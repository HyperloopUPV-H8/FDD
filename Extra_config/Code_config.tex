%---------------------- Colors ----------------------- %

\definecolor{Naranjastring}{HTML}{FF5733}
\definecolor{k3}{HTML}{27A48A}
\definecolor{codegreen}{rgb}{0,0.6,0}
\definecolor{codegray}{rgb}{0.5,0.5,0.5}
\definecolor{codepurple}{rgb}{0.58,0,0.82}
\definecolor{backcolour}{HTML}{F3F8FF}

%---------------------- VDHL ----------------------- %
%Code listing style named "mystyle"
\lstdefinestyle{mystyle}{
  backgroundcolor=\color{backcolour},   
  commentstyle=\color{codegreen},
  keywordstyle=\color{Azul_H8},
  numberstyle=\tiny\color{codegray},
  stringstyle=\color{codepurple},
  basicstyle=\ttfamily\footnotesize,
  breakatwhitespace=false,         
  breaklines=true,                 
  captionpos=b,                    
  keepspaces=true,                 
  numbers=left,                    
  numbersep=5pt,                  
  showspaces=false,                
  showstringspaces=false,
  showtabs=false,                  
  tabsize=2,
  inputpath = codigo,
  postbreak=\mbox{\textcolor{red}{$\hookrightarrow$}\space},
  breaklines = true
  %firstnumber=(auto|last|<number>) para comenzar el conteo de lineas de listing con otro numero. Auto empieza por 1, last, continua la cuenta del listing anterior, number es un numero cualquiera.
}

%---------------------- C ----------------------- %

\lstdefinestyle{myCstyle}{
  backgroundcolor=\color{backcolour},   
  commentstyle=\color{codegreen},
  keywordstyle=\color{coloretsid},
  numberstyle=\tiny\color{codegray},
  stringstyle=\color{red},
  basicstyle=\ttfamily\footnotesize,
  breakatwhitespace=false,         
  breaklines=true,                 
  captionpos=b,                    
  keepspaces=true,                 
  numbers=left,                    
  numbersep=5pt,                  
  showspaces=false,                
  showstringspaces=false,
  showtabs=false,                  
  tabsize=2,
  inputpath = codigo,
  postbreak=\mbox{\textcolor{red}{$\hookrightarrow$}\space},
  breaklines = true,
  keywordstyle=[2]\color{codepurple},
  keywordstyle=[3]\color{k3},
  keywords=[2]{DDRB, DDRC, DDRD, DDB0, DDB1, DDB2, DDB3, DDB4, DDB5, DDB6, DDB7, DDC0, DDC1, DDC2, DDC3, DDC4, DDC5, DDC6, DDC7, DDD0, DDD1, DDD2, DDD3, DDD4, DDD5, DDD6, DDD7, PORTB, PORTC, PORTD, PORTB0, PORTB1, PORTB2, PORTB3, PORTB4, PORTB5, PORTB6, PORTB7,  PORTC0, PORTC1, PORTC2, PORTC3, PORTC4, PORTC5, PORTC6, PORTC7,  PORTD0, PORTD1, PORTD2, PORTD3, PORTD4, PORTD5, PORTD6, PORTD7, PINB, PINC, PIND, PINB0, PINB1, PINB2, PINB3, PINB4, PINB5, PINB6, PINB7, PINC0, PINC1, PINC2, PINC3, PINC4, PINC5, PINC6, PINC7, PIND0, PIND1, PIND2, PIND3, PIND4, PIND5, PIND6, PIND7, ADMUX, REFS0, MUX0, MUX1, MUX2, MUX3, ADCSRA, ADATE, ADPS2, ADPS1, ADPS0, ADEN, ADIE, ADSC, ADC_vect, ADCW, TCCROA, WGM01, OCR0A, TIMSK0, OCIE0A, TCCR0B, CS02, CS01, CS00, TIMER0_COMPA_vect},
  keywords=[3]{main},
  morekeywords = {_delay_ms, sei, ISR, uint8_t, uint16_t}
  %firstnumber=(auto|last|<number>) para comenzar el conteo de lineas de listing con otro numero. Auto empieza por 1, last, continua la cuenta del listing anterior, number es un numero cualquiera.
}

%---------------------- MATLAB ----------------------- %

%Code listing style named "mystyle"
\lstdefinestyle{myMstyle}{
  backgroundcolor=\color{backcolour},   
  commentstyle=\color{codegreen},
  keywordstyle=\color{coloretsid},
  numberstyle=\tiny\color{codegray},
  stringstyle=\color{codepurple},
  basicstyle=\ttfamily\normalsize,
  breakatwhitespace=false,         
  breaklines=true,                 
  captionpos=b,                    
  keepspaces=true,                 
  numbers=left,                    
  numbersep=5pt,                  
  showspaces=false,                
  showstringspaces=false,
  showtabs=false,                  
  tabsize=2,
  inputpath = codigo,
  postbreak=\mbox{\textcolor{red}{$\hookrightarrow$}\space},
  breaklines = true
  keywordstyle=[2]\color{black},
  keywords=[2]{path}
  %firstnumber=(auto|last|<number>) para comenzar el conteo de lineas de listing con otro numero. Auto empieza por 1, last, continua la cuenta del listing anterior, number es un numero cualquiera.
}

%---------------------- SystemVerilog ----------------------- %

%Code listing style named "mystyle"
\lstdefinestyle{svStyle}{
  backgroundcolor=\color{backcolour},   
  commentstyle=\color{codegreen},
  keywordstyle=\color{coloretsid},
  numberstyle=\tiny\color{codegray},
  stringstyle=\color{codepurple},
  basicstyle=\ttfamily\footnotesize,
  breakatwhitespace=false,         
  breaklines=true,                 
  captionpos=b,                    
  keepspaces=true,                 
  numbers=left,                    
  numbersep=5pt,                  
  showspaces=false,                
  showstringspaces=false,
  showtabs=false,                  
  tabsize=2,
  inputpath = codigo,
  postbreak=\mbox{\textcolor{red}{$\hookrightarrow$}\space},
  breaklines = true,
  %keywordstyle=[2]\color{coloretsid},
  keywordstyle=[3]\color{codepurple},
  %keywords=[2]{always_comb, always_ff, always}
  keywords=[3]{logic},
  morekeywords = {always_comb, always_ff, always}
  %firstnumber=(auto|last|<number>) para comenzar el conteo de lineas de listing con otro numero. Auto empieza por 1, last, continua la cuenta del listing anterior, number es un numero cualquiera.
}

%"mystyle" code listing set
\lstset{language = vhdl, style=mystyle}
\lstset{language = C, style=myCstyle}
\lstset{language = matlab, style=myMstyle}
\lstset{language = verilog, style=svStyle}

\renewcommand{\lstlistingname}{Source code}% Listing -> Algorithm
\renewcommand{\lstlistlistingname}{List of \lstlistingname s}% List of Listings -> List of Algorithms