\part*{General}
\chapter*{Description of the Team}


Hyperloop UPV is a student team born and raised in Valencia, Spain. Based in the Universitat Politècnica de València UPV, it has relentlessly devoted eight years to the research and development of hyperloop technology. As a multidisciplinary team, its members embody student representation from all faculties of the University. 

The team structure is based on the principle of ensuring coworking, such that challenges that may arise throughout the year can be tackled critically and creatively from multiple points of view.

The success of this approach largely depends on arranging the members in separate working groups, known as subsystems; each focused on a different aspect of hyperloop development. The team comprises 44 members and 13 collaborators with expertise in varied hyperloop-related areas, from engineering to operations management. Furthermore, the team has been endorsed from its first steps by two faculty advisors: Dr. Vicente Dolz, Ph.D. in Mechanical Engineering, and Dr. Tomás Baviera, Ph.D. in Journalism. 

On top of that, Hyperloop UPV has evolved to be firmly rooted in the training aspect of its members. Not only is it a competitive and internationally leading team, but it is also a valuable source of knowledge for students in a variety of different areas, both those related to the technology itself and those of a more transversal nature. Proof of it is the fact that, over the course of its existence, the team has housed more than 30 students from different degrees who based their thesis on the activity of the team.

A detailed list of team members and collaborators is shown in Table %\ref{tab:s_1.1_list_members}.
\\
TABLA 1.1
\\
The origins of Hyperloop UPV date back to 2015, alongside five students determined to reach one goal: participating in the Hyperloop Design Weekend held by SpaceX. Not only did they achieve that, but on top of it, they were awarded both the ‘Top Design Concept’ and the ‘Propulsion/Compression Subsystem Technical Excellence Award’. The following year, driven by their accomplishments and in cooperation with Purdue University, they built Atlantic II, the first actual vehicle to participate in the ‘Hyperloop Pod Competition II’. It is worth highlighting that it completed every test until the eighth, the Open Air Test in SpaceX Hyperloop. 

In its third year, the team competed with its third vehicle: Turian, which managed to position Hyperloop UPV among the top 8 at the Hyperloop Pod Competition IV. As a result of the improvement demonstrated year by year, the team received, in said edition, the ‘Innovation Award’. 

Naturally, for Hyperloop UPV, 2020 made history in utterly different ways. The team shifted its focus from competing in SpaceX to organizing and competing in a different event: the European Hyperloop Week. The foundation of said transition was the understanding of the year as an opportunity to bet on scalability, expanding and yet deepening in the hyperloop concept, and the thirst for having a real impact on society and the transport industry. Altogether, they led the team to the creation of Ignis, the first-ever vehicle in Spain powered by a Double Linear Induction Motor (DLIM).

Ignis signified an undeniable turning point, driving the approach of the vehicle developed in the last edition, Auran, which stood on the grounds of the knowledge gained about the behaviour of a Linear Induction Motor. Similarly, the core of the present proposal lies in the knowledge gained with Auran, particularly concerning electromagnetic levitation and guiding technology. It was at that point, after a year marked by firsts, and in which the team broke with the norm, that Hyperloop UPV was the most awarded team of the EHW, bringing home four awards: the ‘Thermal Management Award’, the ‘Most Scalable Prototype Award’, the ‘Ingenuity Award’ and the ‘Best Guiding Subsystem Award’. 

The current vision of the team is that of directing the focus towards a technology that is scalable, reliable, and at the same time as optimized as possible. This year, from the very beginning, the mission has been to achieve a complete hyperloop vehicle.

Thus, the result of it all is born, baptised under the name of  \Nombre as the first vehicle in the history of the team to implement hyperloop technology in its entirety, following the current approach. The name seeks to epitomise the very essence of this eighth generation, the fierce commitment to reliability and zero friction -both with the air and any surface. The name is a nod to a god from Selknam mythology, who, standing on the shoulders of giants, shaped the earth just as this vehicle shapes the future. 

Along the same lines, this year, the team has developed an infrastructure based on the one already created last year to adapt to the new challenge: the vacuum environment. The tube is not just a mere mechanical system for Hyperloop UPV, it is the backbone of the project. Following this idea, the team baptises its infrastructure under the name \textbf{Atlas} in reference to the first vertebra of the spinal cord, the one that supports our head. It is the first, the most important, and distinct from all the others; that is why it alone can withstand our world. This vertebra is named after the Greek titan whom Zeus condemned to hold the world on his shoulders for eternity, tirelessly.\\ 

%As such, \nombre has created the world of this eighth generation, and it is \textbf{Atlas} that holds it. 

In Figure %\ref{tab:s_1.1_pods_timeline}, a brief timeline gathering all the prototypes throughout the history of the team until the present day is depicted. 

\\
%igura 1.1

As this document unfolds, the functioning of \Nombre will be thoroughly detailed, including an in-depth description of the design process, which conditions the manufacturing, assembly and testing of both the vehicle and its infrastructure.